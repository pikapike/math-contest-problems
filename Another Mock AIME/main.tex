\documentclass{article}
\usepackage[utf8]{inputenc}

\usepackage{amsmath}

\begin{document}

% \iffalse

\begin{enumerate} 

% geometry
\item Two circles $\omega_1$ and $\omega_2$, each of radius 112, are drawn such that their centers lie on each others’ circumferences. Then, two smaller circles with equal radii are drawn such that they are both internally tangent to $\omega_1$ and $\omega_2$, and are externally tangent to each other. What is the radius of one of the smaller circles?

% algebra
\item Each of the functions $f(x)$ and $g(x)$ is either equal to $\sqrt{x-1}+\sqrt{x-9}$ or $\sqrt{x-1}-\sqrt{x-9}$ (the functions may be the same or different). The equation $f(g(x))=2$ has exactly one real solution, which can be expressed as $\frac{p}{q}$, where $p$ and $q$ are relatively prime positive integers. Compute $p+q$.

% number theory
\item Noelle writes down the factors of $2^4 \cdot 17^9$ from least to greatest. Starting with the first factor, she circles every other factor. Noelle then records the sum of all of the circled factors, as well as sum of all of the uncircled factors. The ratio of the smaller recorded sum to the larger recorded sum is $\frac{a}{b}$, where $a$ and $b$ are relatively prime positive integers. Compute $2a+b$.

% geometry
\item Point $P$ is located in the interior of regular hexagon $ABCDEF$ so that the distance from $P$ to the sides $AB$, $CD$, and $EF$ are 8, 4, and 9 respectively. Then the length of $AB$ is $\frac{a\sqrt{b}}{c}$, where $a$ and $c$ are relatively prime and $b$ is not divisible by the square of any prime. Find $a+b+c$.

% combo
\item A token is placed in the top left square of a $5 \times 4$ checkerboard. A move consists of moving the token one square up, down, to the left, or to the right, so long as the token does not move off of the checkerboard. How many nine-move sequences move the token to the bottom right square?

% number theory
\item Define the sequence $\{ a_n \}$ so that $a_0=1, a_1=2, a_2=3$, and for all $n>2$, \[ a_n= \mathrm{lcm}(a_{n-1},a_{n-2}+a_{n-3}). \]
Compute the remainder when the number of divisors of $a_{100}$ is divided by 1000.

% number theory
\item Suppose $a,b,c$ and $d$ are (not necessarily distinct) prime numbers such that $a+b+c-3d=4^2$ and $ab+bc+ca-3d=18^2$. Compute $abc-3d$.

% combo
\item Seven teams play a round-robin tournament, where each pair of teams plays each other exactly once. In any match, the two teams are equally likely to win, with no ties, and all match results are independent. The probability that at most five teams win at least two matches is $\frac{p}{q}$, where $p$ and $q$ are relatively prime positive integers. Find $p$.

% geometry
\item A spotlight is 30 meters away from a very large wall, and can cast a beam of light in the shape of a right circular cone, with its apex at the spotlight's location. The region of the wall that the spotlight illuminates is a circular disk with center $A$ and an area of $300\pi$ square meters. The spotlight is then swiveled so that point $A$ lies on the perimeter of the illuminated area. The new illuminated area, in square meters, is $a\pi \sqrt{b}$, where $b$ is not divisible by the square of any prime. Compute $a+b$.

% combo
\item Delthea brings two copies each of four different books to a book club and gives them to five friends so that no one gets two copies of the same book. She then observes that the only non-empty subset of friends for which the books she gave them can be paired up with their identical copies is the entire group of five friends. If $N$ is the number of ways that Delthea could have given the books, find $N/10$.

% algebra
\item Let $P(x) = a_kx^k + a_{k-1}x^{k-1} + \ldots + a_1x + a_0$ be the polynomial that satisfies $(x^9+x-1)P(x) = (x^9-1)^9-x+1$ for all $x$. Compute $\displaystyle \sum_{i=0}^k |a_i|$.

% geometry/algebra - make sure this is a possible configuration?
\item Points $A,B,C,D,E,$ and $F$ lie on a circle, in that order. The region enclosed by the chords $AD, BE,$ and $CF$ is an equilateral triangle. If $AC^2+CE^2+EA^2=475$ and $[\triangle ACE]-[\triangle BDF]=4\sqrt{3}$, find the value of $BD^2+DF^2+FB^2$. (Here $[K]$ denotes the area of the region $K$.)

% number theory
\item Find the least positive integer $n$ such that $126^n-5^n+11n^2$ is divisible by $55^3$.

% combo
\item Start with the number 1, and perform a sequence of moves to turn it into the number 100. A move can be either of these operations:

\begin{itemize}
    \item Add 1 to the number.
    \item If the number has two digits, swap the digits. This may only be done if doing so would increase the value of the number.
\end{itemize}

Let $N$ be the sum of the number of digit swaps that happen, over all different sequences of moves. Compute the remainder when $N$ is divided by 1000. 

% geometry
\item Let $S$ be the set of all complex numbers $w$ such that $|w|=53$. Let $a_1$ and $a_2$ be two distinct complex numbers in $S$ such that $ \displaystyle \frac{a_j-(7+13i)}{a_j-(55-7i)} $ is imaginary for $j=1,2$, and let $b_1$ and $b_2$ be two distinct complex numbers in $S$ such that $ \displaystyle \frac{b_k-(41+27i)}{b_k-(81+39i)} $  is imaginary for $k=1,2$. Let $z$ be the complex number that satisfies the property that both $\displaystyle \frac{z-a_1}{z-a_2}$ and $\displaystyle \frac{z-b_1}{z-b_2}$ are real. Then $z=m+ni$ for real numbers $m$ and $n$. Find $m+n$.

\end{enumerate}





\newpage

\;

\newpage

Answer key and solutions next page

\newpage

% \fi

%------------------------------%

Answer key:

%------------------------------%

\begin{enumerate}

\item 42

\item 779 ($p=754, q=25$)

\item 749 ($a=191, b=367$)

\item 20 ($a=14, b=3, c=3$)

\item 714

\item 876 ($a_{100} = 2^{48} \cdot 3^{17} \cdot 5^{17}$)

\item 498 ($\{ a,b,c \} = \{ 3,3,61 \}, d=17$)

\item 973 ($p=973, q=2^{13}$)

\item 231 ($a=225, b=6$)

\item 300

\item 510

\item 523

\item 528

\item 911 ($N = 5911$)

\item 68

\end{enumerate}

\newpage

%------------------------------%

Solutions:

%------------------------------%

\begin{enumerate}

\item Two circles $\omega_1$ and $\omega_2$, each of radius 112, are drawn such that their centers lie on each others’ circumferences. Then, two smaller circles with equal radii are drawn such that they are both internally tangent to $\omega_1$ and $\omega_2$, and are externally tangent to each other. What is the length of the radius of one of the smaller circles?

\textbf{Answer (42):} Let the centers of $\omega_1$ and $\omega_2$ be $A$ and $B$, and let the center of a smaller circle be $O$. Using symmetry, $AB$ is tangent to both of the smaller circles at its midpoint, which we will call $M$. Let the radius of the smaller circles be $r$, and let $P$ be the point where the smaller circle with center $O$ is internally tangent to the circle with center $A$. Then $AM = \frac{112}{2} = 56$, $OM = r$, and $OA = PA-PO = 112-r$. Since $\angle AMO$ is right, by the Pythagorean theorem $AM^2+OM^2=OA^2$, so: \[ 56^2+r^2=(112-r)^2 \Rightarrow 56^2 = 112^2-2\cdot 112r \Rightarrow r= \frac{112^2-56^2}{2 \cdot 112} = \boxed{42} \] 

\item Each of the functions $f(x)$ and $g(x)$ is either equal to $\sqrt{x-1}+\sqrt{x-9}$ or $\sqrt{x-1}-\sqrt{x-9}$. The equation $f(g(x))=2$ has exactly one real solution, which can be expressed as $\frac{p}{q}$, where $p$ and $q$ are relatively prime positive integers. Compute $p+q$.

\textbf{Answer (779):} First, let's solve for the value of $g(x)$. Since $f(g(x))=2$, either $\sqrt{g(x)-1}+\sqrt{g(x)-9}=2$ or $\sqrt{g(x)-1}-\sqrt{g(x)-9}=2$. However, in both cases it must be that $g(x) \ge 3$, so $\sqrt{g(x)-1}+\sqrt{g(x)-9} \ge \sqrt{9-1}+\sqrt{9-9} = 2\sqrt{2} > 2$, rendering the first case impossible. Therefore, $\sqrt{g(x)-1}-\sqrt{g(x)-9}=2$, and by inspection and the fact that the function $\sqrt{x-1}-\sqrt{x-9}$ is strictly decreasing, $g(x)=10$ is the only solution.

Now, either $\sqrt{x-1}+\sqrt{x-9}=10$ or $\sqrt{x-1}-\sqrt{x-9}=10$. Since the maximum value of $\sqrt{x-1}-\sqrt{x-9}$ is $2\sqrt{2}$, which is less than $10$, the second case is impossible, so $\sqrt{x-1}+\sqrt{x-9}=10$. We now solve for $x$, starting by squaring both sides:

\begin{align*}
    &(\sqrt{x-1}+\sqrt{x-9})^2 = (x-1)+(x-9)+2\sqrt{(x-1)(x-9)} = 100 \\
    &\Rightarrow \sqrt{x^2-10x+9}=55-x \\
    &\Rightarrow x^2-10x+9=(110-2x)^2=55^2-110x+x^2 \\
    &\Rightarrow 100x=55^2-9=58\cdot 52 \Rightarrow x=\frac{58\cdot 13}{25}=\frac{754}{25}
\end{align*}

Thus $p+q=754+25=\boxed{779}$.

\newpage

\item Noelle writes down the factors of $2^4 \cdot 17^9$ from least to greatest. Starting with the first factor, she circles every other factor. Noelle then records the sum of all of the circled factors, as well as sum of all of the uncircled factors. The ratio of the smaller recorded sum to the larger recorded sum is $\frac{a}{b}$, where $a$ and $b$ are relatively prime positive integers. Compute $2a+b$.

\textbf{Answer (749):} Since $2^4 < 17$, the factors of $2^4 \cdot 17^9$ when written in order will be the factors of $2^4$, then the factors of $2^4$ multiplied by $17^k$ for each successive positive integer $k$ up to 10. Written in a chart-like array, with every other factor circled, the factors look like:

\[ 
\begin{matrix}
    \boxed{2^0} & 2^1 & \boxed{2^2} & 2^3 & \boxed{2^4} \\
    2^0 \cdot 17 & \boxed{2^1 \cdot 17} & 2^2 \cdot 17 & \boxed{2^3 \cdot 17} & 2^4 \cdot 17 \\
    \boxed{2^0 \cdot 17^2} & 2^1 \cdot 17^2 & \boxed{2^2 \cdot 17^2} & 2^3 \cdot 17^2 & \boxed{2^4 \cdot 17^2} \\
    2^0 \cdot 17^3 & \boxed{2^1 \cdot 17^3} & 2^2 \cdot 17^3 & \boxed{2^3 \cdot 17^3} & 2^4 \cdot 17^3 \\
    \vdots & \vdots & \vdots & \vdots & \vdots \\
\end{matrix}
\]

Note that each group of every two rows repeats, except that all terms are multiplied by $17^2$ each time. So the ratio of the sum of circled terms to uncircled terms is the same as the ratio computed just for the first two rows, which is $\frac{2^0+2^2+2^4+2^1 \cdot 17+2^3 \cdot 17}{2^1+2^3+2^0 \cdot 17+2^2 \cdot 17+2^4 \cdot 17} = \frac{191}{367}$. Therefore, $2a+b=2 \cdot 191+367=\boxed{749}$.

\item Point $P$ is located in the interior of regular hexagon $ABCDEF$ so that the distance from $P$ to the sides $AB$, $CD$, and $EF$ are 8, 4, and 9 respectively. Then the length of $AB$ is $\frac{a\sqrt{b}}{c}$, where $a$ and $c$ are relatively prime positive integers and $b$ is not divisible by the square of any prime. Find $a+b+c$.

\textbf{Answer (20):} Let $AB = s$, and let the lines $AB$ and $EF$ intersect at $X$, lines $AB$ and $CD$ intersect at $Y$, and lines $CD$ and $EF$ intersect at $Z$. Then $\triangle XYZ$ is equilateral, and has side length $XY = XZ = YZ = 3s$. Note that $[PQR] = \frac{(3s)^2\sqrt{3}}{4} = \frac{9s^2\sqrt{3}}{4}$. Since we know the lengths of the altitudes from $P$ to the side lengths of $\triangle XYZ$, we can write the area of $\triangle XYZ$ in another way: $[XYZ] = [XPY]+[YPZ]+[ZPX] = \frac{1}{2}(XY)(8) + \frac{1}{2}(YZ)(4) + \frac{1}{2}(ZX)(9) = \frac{21}{2}(3s)$. Setting the two expressions equal, we get: \[ \frac{9s^2\sqrt{3}}{4} = \frac{21}{2}(3s) \Rightarrow s = \frac{14\sqrt{3}}{3}\] So, $a+b+c = 14+3+3 = \boxed{20}$.

\newpage

\item A token is placed in the top left square of a $5 \times 4$ checkerboard. A move consists of moving the token one square up, down, to the left, or to the right, so long as the token does not move off of the checkerboard. How many nine-move sequences move the token to the bottom right square?

\textbf{Answer (714):} Let $U$, $D$, $L$, and $R$ denote a move upwards, downwards, leftwards, and rightwards. A seven-move path from the top left corner to the bottom right corner would be a permutation of $DDDRRRR$; an nine-move sequence would have an additional $UD$ or a $LR$ mixed in, with some constraints. For the first case, where $UD$ is added in to $DDDRRRR$, the $U$ must not be before or after all of the $D$s, as that would mean the token moves off of the grid at some point. So we first permute $DDDDDRRRR$, and then change one of the middle $D$s to a $U$; there are $\binom{9}{4} \cdot 3 = 378$ ways to do this. Similarly, for the second case where $LR$ is added into $DDDRRRR$, the $L$ cannot go before or after all of the $R$s, so we first permute $DDDRRRRRR$, and then change one of the middle $R$s to an $L$; there are $\binom{9}{3} \cdot 4 = 336$ ways in this case. The total number of sequences of moves is $378+336=\boxed{714}$.

\item Define the sequence $a_n$ so that $a_0=1, a_1=2, a_2=3$, and for all $n>2$, \[ a_n= \mathrm{lcm}(a_{n-1},a_{n-2}+a_{n-3}). \]
Compute the remainder when the number of divisors of $a_{100}$ is divided by 1000.

\textbf{Answer (876):} We compute the first few terms of the sequence:

\begin{table}[h!]
\centering
\begin{tabular}{|l|l|l|l|l|l|l|l|l|l|l|l|}
\hline
n  & 0 & 1 & \textbf{2} & \textbf{3} & \textbf{4} & 5 & 6  \\ \hline
$a_n$ & 1 & 2 & \textbf{3} & \textbf{3} & $\mathbf{3 \cdot 5}$ & $2 \cdot 3 \cdot 5$ & $2 \cdot 3^2 \cdot 5$ \\ \hline
\end{tabular}

\begin{tabular}{|l|l|l|l|l|l|l|l|l|l|l|l|}
\hline
n  & 7 & \textbf{8} & \textbf{9} & \textbf{10} \\ \hline
$a_n$ & $2 \cdot 3^2 \cdot 5$ & $\mathbf{2^3 \cdot 3^2 \cdot 5}$ & $\mathbf{2^3 \cdot 3^2 \cdot 5}$ & $\mathbf{2^3 \cdot 3^2 \cdot 5^2}$ \\ \hline
\end{tabular}
\end{table}

Note that the numbers in bold are three consecutive terms that are repeated with a gap of 6, except all of them are multiplied by $2^3 \cdot 3 \cdot 5$. Using the fact that $\mathrm{lcm}(km, kn) = k\mathrm{lcm}(m,n)$ for all integer $k,m,n$, this pattern will continue forever, so $a_{n+6} = 2^3 \cdot 3 \cdot 5 \cdot a_n$ for all $n \ge 2$. Therefore: \[ a_{100} = a_{4 + 16 \cdot 6} = (2^3 \cdot 3 \cdot 5)^{16} \cdot a_4 = 2^{48} \cdot 3^{17} \cdot 5^{17} \] The number of divisors of $a_{100}$ is then $(48+1)(17+1)(17+1) = 15\boxed{876}$.

\newpage

\item Suppose $a,b,c$ and $d$ are (not necessarily distinct) prime numbers such that $a+b+c-3d=4^2$ and $ab+bc+ca-3d=18^2$. Compute $abc-3d$.

\textbf{Answer (498):} Since both equations have a $-3d$ that otherwise seems hard to use, we consider taking the equations modulo 3 to get rid of it. We have: \begin{align*}
    a+b+c \equiv 4^2 \equiv 1 \bmod 3 \\
    ab+bc+ca \equiv 18^2 \equiv 0 \bmod 3 
\end{align*}

Looking at the second equation, the possibilities for the moduli of $ab$, $bc$, and $ca$ are limited: $(ab,bc,ca) \equiv (0,0,0),(1,1,1),(2,2,2),(0,1,2)$ or permutations thereof. The case $(0,1,2)$ is impossible, as $a,b,$ or $c$ would be $0 \bmod 3$, but that would mean two of $ab$, $bc$, and $ca$ would be $0 \bmod 3$. For $(2,2,2)$, each of the pairs $(a,b)$, $(b,c)$, and $(c,a)$ must have one $1 \bmod 3$ element and one $2 \bmod 3$ element, which is an impossible task. For $(1,1,1)$, it must be that $a \equiv b \equiv c \bmod 3$ (and of nonzero modulus), but that means $a+b+c \equiv 0 \not\equiv 1 \bmod 3$. We are left with $(ab,bc,ca)=(0,0,0)$, in which case without loss of generality, $a \equiv b \equiv 0 \bmod 3$ and $c \equiv 1 \bmod 3$.

Since $a \equiv b \equiv 0 \bmod 3$ and $a$ and $b$ are prime, $a = b = 3$. Substituting into the given equations yields a system of equations in two variables: $6+c-3d=16$ and $9+6c-3d=324$. Solving, we get $c = 61$ and $d = 17$ (check: $c$ and $d$ are both prime!). To finish, $abc-3d = 3 \cdot 3 \cdot 61 - 3 \cdot 17 = \boxed{498}$.

\item Seven teams play a round-robin tournament, where each pair of teams plays each other exactly once. In any match, the two teams are equally likely to win, with no ties, and all match results are independent. The probability that at most five teams win at least two matches is $\frac{p}{q}$, where $p$ and $q$ are relatively prime positive integers. Find $p$.

\textbf{Answer (973):} More simply put, we wish to find the probability that at least two teams win at most one match each. There are two main cases:

\textbf{Case 1:} Team A and Team B lose all games except their head-to-head matchup (where one of them wins their one game, and the other loses all games). There are $\binom{7}{2}$ ways to choose two teams, and the probability that they each lose all of their games against the other five teams is $\frac{1}{2^5} \cdot \frac{1}{2^5} = \frac{1}{2^{10}}$, so this case has a probability of $\binom{7}{2} \cdot \frac{1}{2^{10}} = \frac{21}{2^{10}}$ of occurring.

\textbf{Case 2:} Team A,B,C beat each other in a cyclical fashion (A $\rightarrow$ B $\rightarrow$ C $\rightarrow$ A or A $\rightarrow$ C $\rightarrow$ B $\rightarrow$ A), and at least two of these teams lose all other games. There are $\binom{7}{3}$ ways to choose three teams, and the probability that they beat each other in a cyclical fashion is $\frac{1}{4}$. There are 3 ways to choose which two teams lose all other games; given the choice, the possibility then happens with probability $\frac{1}{2^4} \cdot \frac{1}{2^4} = \frac{1}{2^8}$. But we count the case where all three teams lose all of their other games three times, overcounting this possibility twice, so we subtract off $2 \cdot \frac{1}{2^4} \cdot \frac{1}{2^4} \cdot \frac{1}{2^4} = \frac{1}{2^{11}}$ to get that the probability that at least two of the three teams lose all of their other games is $3 \cdot \frac{1}{2^8}-\frac{1}{2^{11}} = \frac{23}{2^{11}}$. In all, the probability that this case occurs is $\binom{7}{3} \cdot \frac{1}{4} \cdot \frac{23}{2^{11}} = \frac{805}{2^{13}}$.

The total probability is $\frac{21}{2^{10}} + \frac{805}{2^{13}} = \frac{973}{2^{13}}$, and the numerator is $\boxed{973}$.

\item A spotlight is 30 meters away from a very large wall, and can cast a beam of light in the shape of a right circular cone, with its apex at the spotlight's location. The region of the wall that the spotlight illuminates is a circular disk with center $A$ and an area of $300\pi$ square meters. The spotlight is then swiveled so that point $A$ lies on the perimeter of the illuminated area. The new illuminated area, in square meters, is $a\pi \sqrt{b}$, where $b$ is not divisible by the square of any prime. Compute $a+b$.

\textbf{Answer (231):} The initial circular disk, which we'll call $\gamma$ for future reference, has radius $10\sqrt{3}$, and since the height, slant height, and radius of the disk create a 30-60-90 triangle, the angle that the height and slant height make in the cone is $30^{\circ}$. When the spotlight is swiveled, the new illuminated region is an ellipse, since the intersection of a cone is always a conic. We now proceed to find the major and minor axes of the ellipse. Let $P$ be the vertex of the cone, let $AB$ be the major axis of the ellipse, let $CD$ be the minor axis of the ellipse, let $Q$ be the intersection of $\gamma$ with $AB$, and let $R$ be on $\gamma$ such that $QR$ is a diameter of $\gamma$.

If we take the cross-section of the cone using the plane perpendicular to the wall and containing the line $\overline{AB}$, we can see that $\angle PAB = 30^{\circ}+30^{\circ}=60^{\circ}$, so $\triangle BAP$ is a 30-60-90 triangle. Therefore, $AB = AP\sqrt{3} = 30\sqrt{3}$, and the major axis is $\frac{1}{2} \cdot 30\sqrt{3} = 15\sqrt{3}$.

To find the minor axis, draw $\triangle CPD$. Since $CP = PD$ and $\angle CPD = 30^{\circ}+30^{\circ}=60^{\circ}$, $\triangle CPD \sim \triangle QPR$. So, the ratio of the heights from point $P$ is the equal to $\frac{CD}{QR}$. If $M$ is on $CD$ so that $PM$ is an altitude, then $M$ is the center of the ellipse, so $MA = 15\sqrt{2}$, and by the Pythagorean theorem: \[ PM=\sqrt{MA^2+PA^2}=\sqrt{(15\sqrt{2})^2+30^2}=15\sqrt{6} \] So, $CD = QR \cdot \frac{PM}{PA} = 20\sqrt{3} \cdot \frac{15\sqrt{6}}{30} = 30\sqrt{2}$, and the minor axis is $\frac{1}{2} \cdot 30\sqrt{2} = 15\sqrt{2}$.

We conclude that the area of the ellipse is $(15\sqrt{3})(15\sqrt{2})\pi = 225\pi\sqrt{6}$, so $a+b=225+6=\boxed{231}$.

\item Delthea brings two copies each of four different books to a book club and gives them to five friends so that no one gets two copies of the same book. She then observes that the only non-empty subset of friends for which the books she gave them can be paired up with their identical copies is the entire group of five friends. If $N$ is the number of ways that Delthea could have given the books, find $N/10$.

\newpage

\textbf{Answer (300):} Since these books come in identical pairs, we may visualize the books as edges of different colors, and the five friends as vertices. Then the given condition ``the only non-empty subset of friends for which the books she gave them can be paired up with their identical copies is the entire group of five friends" simply becomes ``the graph is connected". However, there are only four edges, so the graph must be a tree. It is a known fact that there are $n^{n-2}$ trees on $n$ vertices, so there are $5^{5-2} = 125$ trees in this case (but if one does not know this formula, the number of trees on 5 vertices can be found with a small amount of casework). Lastly, there are $4!=24$ ways to color the edges to ascertain which book copies were given to which friends, so the total number of ways Delthea could have given the books is $125 \cdot 24 = 3000$. To finish, $N/10 = 3000/10 = \boxed{300}$.

\item Let $P(x) = a_kx^k + a_{k-1}x^{k-1} + \ldots + a_1x + a_0$ be the polynomial that satisfies $(x^9+x-1)P(x) = (x^9-1)^9-x+1$ for all $x$. Compute $\displaystyle \sum_{i=0}^k |a_i|$.

\textbf{Answer (510):} To find $P(x)$, we divide $(x^9-1)^9-x+1$ by $x^9+x-1$. We have $(x^9-1)^9 = (x^9+x-1)(x^9-1)^8 - x(x^9-1)^8$, and to repeat the process in general, $x^{9-k}(x^9-1)^k = (x^9+x-1)(x^{9-k})(x^9-1)^{k-1} - x^{9-k}(x^9-1)^{k-1}$. Applying this rule repeatedly to the remainders, the signs alternate while the power of $x$ increases by one each time, resulting in: 

\begin{align*}
    (x^9-1)^9-x+1 &= (x^9+x-1)((x^9-1)^8-x(x^9-1)^7+x^2(x^9-1)^6 \\
    &-\ldots-x^7(x^9-1)+x^8) - x^9 - x+1 \\
    &= (x^9+x-1)((x^9-1)^8-x(x^9-1)^7+x^2(x^9-1)^6 \\
    &-\ldots-x^7(x^9-1)+x^8-1) \\
    \Rightarrow P(x) = (x^9&-1)^8-x(x^9-1)^7+\ldots-x^7(x^9-1)+x^8-1
\end{align*}

To find the sum of the absolute values of the coefficients of $P(x)$, we can observe that none of the expansions of the binomial expressions will overlap, due to each one taking a different modulus modulo 9 for the degrees of their terms, so we can take the sums of the absolute values of the coefficients of the binomial expressions separately. The only expressions that overlap are $(x^9-1)^8$ and $-1$, where the constant coefficients cancel, so we will subtract 1 from the total of the binomial coefficient sum. Note that the sum of the absolute values of $(x^9-1)^k$ is equal to the sum of the coefficients of $(x^9+1)^k$, which upon plugging in $x=1$ is just $2^k$. So the sum of the absolute values of the coefficients of $P(x)$ is $2^8+2^7+2^6+\ldots+2^1+2^0-1=(2^9-1)-1=\boxed{510}$.

\newpage

\item Points $A,B,C,D,E,$ and $F$ lie on a circle, in that order. The region enclosed by the chords $AD, BE,$ and $CF$ is an equilateral triangle. If $AC^2+CE^2+EA^2=475$ and $[\triangle ACE]-[\triangle BDF]=4\sqrt{3}$, find the value of $BD^2+DF^2+FB^2$. (Here $[K]$ denotes the area of the region $K$.)

\textbf{Answer (523):} Let $P$ be the intersection of $AD$ and $BE$, let $Q$ be the intersection of $AD$ and $CF$, and let $R$ be the intersection of $BE$ and $CF$. Then $\triangle PQR$ is an equilateral as given in the problem; let $PQ = QR = RP = s$. Let $AP = a$, $BP = b$, $CQ = c$, $DQ = d$, $ER = e$, and $FR = f$, and assume that none of these segments contain a side of $\triangle PQR$. (If this isn't the case, then we can just reassign the points $B \leftrightarrow E$ and $C \leftrightarrow F$ so that it works out while all given equations remain the same.)

Now, by Power of a Point on each pair of chords: \begin{align*}
    a(s+d) &= b(s+e) \\
    c(s+f) &= d(s+a) \\
    e(s+b) &= f(s+c) 
\end{align*}

Adding all three equations together yields $s(a+c+e)+ad+cf+eb=s(b+d+f)+be+da+fc$, and simplifying, $a+c+e=b+d+f$.

Turning our attention to the second equation, we split the areas into multiple triangles and use the sine area formula with the $120^{\circ}$ angles: \begin{align*}
    4\sqrt{3} = [ACE] - [BDF] &= ([AQC]+[CRE]+[EPA]+[PQR]) \\ 
    &-([BPD]+[DQF]+[FRB]+[PQR]) \\
    &= \frac{\sqrt{3}}{4} \cdot ((ac+ce+ea)-(bd-df-fb)) \\
    &\Rightarrow ac+ce+ea = bd+df+fb+16
\end{align*}

For the first equation, we use the law of cosines with the $120^{\circ}$ angles: \begin{align*}
    &475 = AC^2+CE^2+EA^2 \\
    &= (a+s)^2+(a+s)c+c^2+(c+s)^2+(c+s)e+e^2 \\ 
    &\; \; \; \; +(e+s)^2+(e+s)a+a^2 \\
    &= 2(a^2+c^2+e^2)+(ac+ce+ea)+s(a+c+e)+3s^2\\
    &= 2(a+c+e)^2-3(ac+ce+ea)+s(a+c+e)+3s^2 \\
    &= 2(b+d+f)^2-3(bd+df+fb+16)+s(b+d+f)+3s^2 \\
    &= 2(b+d+f)^2-3(bd+df+fb)+s(b+d+f)+3s^2-48 \\
    &= BD^2+DF^2+FB^2-48
\end{align*}

We conclude that $BD^2+DF^2+FB^2 = 475+48 = \boxed{523}$.

% ((a+s)^2+(a+s)c+c^2+(c+s)^2+(c+s)e+e^2+(e+s)^2+(e+s))

\newpage

\item Find the least positive integer $n$ such that $126^n-5^n+11n^2$ is divisible by $55^3$.

\textbf{Answer (528):} We can solve the modular equations $126^n-5^n+11n^2 \equiv 0 \mod 11^3$ and $126^n-5^n+11n^2 \equiv 0 \mod 5^3$ separately, and combine them afterwards by the Chinese Remainder Theorem.

First, we solve $126^n-5^n+11n^2 \equiv 0 \mod 11^3$. Since $126-5=121=11^2$, we can use the LTE lemma, which yields $\upsilon_{11}(126^n-5^n)=\upsilon_{11}(126-5)+\upsilon_{11}(n)=2+\upsilon_{11}(n)$. This means that if $11 | n$, then $11^3 | 126^n-5^n$, but we also have $11^3 | 11n^2$ at the same time, so $n \equiv 0 \bmod 11$ are solutions to $126^n-5^n+11n^2 \equiv 0 \bmod 11^3$. Since we always have $11^2 | 126^n-5^n$, and $11^2 | 11n^2$ if and only if $11 | n$, $11^2 | 126^n-5^n+11n^2$ if and only if $11 | n$, and therefore these are the only solutions.

Next, we solve $126^n-5^n+11n^2 \equiv 1+11n^2 \equiv 0 \mod 5^3$. Solving for $n^2$ gives $n^2 \equiv (-11)^{-1} \equiv 34 \bmod 5^3$. Taking the equation modulo 5 gives $n^2 \equiv 4 \bmod 5$, or $n \equiv 2,3 \bmod 5$, so either $n = 5a+2$ or $n = 5a+3$ for some integer $a$. 

For the case $n = 5a+2$, plugging into $n^2 \equiv 34 \bmod 125$ gives $(5a+2)^2 \equiv 25a^2+20a+4 \equiv 34 \bmod 5^3$, or $5a^2+4a \equiv 6 \bmod 5^2$; taking this equation modulo 5 gives $4a \equiv 1 \bmod 5$, so $a \equiv 4 \bmod 5$. Write $a = 5b+4$ for some integer $b$. Substituting into $5a^2+4a \equiv 6 \bmod 5^2$ gives $5(5b+4)^2+4(5b+4) \equiv 20b+21 \equiv 6 \bmod 5^2$, or $4b \equiv 2 \bmod 5$, so $b \equiv 3 \bmod 5$. Upon writing $b = 5c+3$ for some integer $c$, we get $n = 125c+97$, so $n \equiv 97 \bmod 125$.

For the case $n = 5a+3$, plugging into $n^2 \equiv 34 \bmod 125$ gives $(5a+3)^2 \equiv 25a^2+30a+9 \equiv 34 \bmod 5^3$, or $5a^2+6a \equiv 5 \bmod 5^2$; taking this equation modulo 5 gives $a \equiv 0 \bmod 5$. Write $a = 5b$ for some integer $b$. Substituting into $5a^2+6a \equiv 5 \bmod 5^2$ gives $5(5b)^2+6(5b) \equiv 5b \equiv 5 \bmod 5^2$, or $b \equiv 1 \bmod 5$. Upon writing $b = 5c+1$ for some integer $c$, we get $n = 125c+28$, so $n \equiv 28 \bmod 125$.

Finally, we want to find the least value of $n$ such that $11 | n$ and either $n \equiv 28 \bmod 125$ or $n \equiv 97 \bmod 125$. The first few positive integers that are $28 \bmod 125$ are $28, 153, 278, 403, 528, \ldots$, where $528$ is a multiple of 11. The first few positive integers that are $97 \bmod 125$ are $97, 222, 347, 472, 597, \ldots$, none of which are a multiple of 11. Therefore, the least possible value of $n$ is $\boxed{528}$.

\item Start with the number 1, and perform a sequence of moves to turn it into the number 100. A move can be either of these operations:

\begin{itemize}
    \item Add 1 to the number.
    \item If the number has two digits, swap the digits. This may only be done if doing so would increase the value of the number.
\end{itemize}

Let $N$ be the sum of the number of digit swaps that happen, over all different sequences of moves. Compute the remainder when $N$ is divided by 1000. 

\textbf{Answer (911):} Some first initial observations: since the only way to go from 1 to 11 is adding 1, and likewise for 99 to 100, we can start at 11 and end at 99. Also, we really only need to focus on swap move, as the add move fills out the space between. So, drawing a grid where each row is $\overline{m0}, \overline{m1}, \ldots, \overline{m9}$ and each column is $\overline{0n}, \overline{1n}, \ldots, \overline{9n}$, we can see that the sequences of moves going from $\overline{aa}$ to $\overline{(a+k)(a+k)}$ are in bijection with the sequences of moves going from $\overline{bb}$ to $\overline{(b+k)(b+k)}$, just by shifting the moves so that the starting point moves from $\overline{aa}$ to $\overline{bb}$. Inspired by this, we find a recursion to work out the number of sequences going from $11$ to $\overline{nn}$ for $n=1,2,3,\ldots,8$ which we'll call $A_n$, so that we can compute the number of moves that use each possible swap move.

The base case is, of course, $A_1 = 1$. To find an expression for $A_n$, we will consider the use of the swap moves $\overline{an} \rightarrow \overline{na}$ for $1 \le a < n$. If none of them are used, then the number of sequences is $A_{n-1}$: we go from $11$ to $\overline{(n-1)(n-1)}$ without having any of the new swap moves, and then repeat adding 1 for the rest of the way. At most one of the swap moves can be used, so let $\overline{an} \rightarrow \overline{na}$ be used for some $a$; then the number of sequences using this swap move is $A_a$: we go from $11$ to $\overline{aa}$, then add 1 repeatedly until we get to $\overline{an}$, then use the swap move $\overline{na}$, and are forced to keep adding 1 the rest of the way. So our recursion is $A_n = A_{n-1}+(A_{n-1}+A_{n-2}+\ldots+A_1) = 2A_{n-1}+A_{n-2}+A_{n-3}+\ldots+A_1$. 

Writing out the first few terms of the recursion, $A_2 = 2$, $A_3 = 5$, and $A_4 = 13$, at which point we can guess that $A_n = F_{2n-1}$, where $F_n$ is the Fibonacci sequence ($F_1=F_2=1$, $F_n=F_{n-1}+F_{n-2}$). This can be proven by induction and the identity $F_{2k} = F_{2k-1}+F_{2k-3}+\ldots+F_1$.

Now let's count the number of times each swap move is used. The swap move $\overline{ab} \rightarrow \overline{ba}$, for $1 \le a < b \le 9$, is used $A_aA_{10-b}$ times: we go from $11$ to $\overline{aa}$, then repeatedly add 1 until $\overline{ab}$, use the swap move $\overline{ab} \rightarrow \overline{ba}$, then repeatedly add 1 until $\overline{bb}$, and finally go from $\overline{bb}$ to $99$. We sum all of these up to get our answer; using the identity $F_{2k} = F_{2k-1}+F_{2k-3}+\ldots+F_1$ again, the total number of times that swap moves are used over all possible sequences of moves is:

\begin{align*}
    \sum_{a=1}^8 \sum_{b=a+1}^9 A_aA_{10-b} &= \sum_{a=1}^8 \sum_{b=1}^{9-a} A_aA_b = \sum_{a=1}^8 F_{2a-1} \sum_{b=1}^{9-a} F_{2b-1} \\
    &= \sum_{a=1}^8 F_{2a-1}F_{2(9-a)} = \sum_{a=1}^8 F_{2a-1}F_{18-2a} \\ 
    = F_{1}F_{16}+F_{3}F_{14}+F_{5}F_{12}&+F_{7}F_{10}+F_{9}F_{8}+F_{11}F_{6}+F_{13}F_{4}+F_{15}F_{2} \\
    = 1 \cdot 987 + 2 \cdot 377 + 5 \cdot 144 &+ 13 \cdot 55 + 34 \cdot 21 + 89 \cdot 8 + 233 \cdot 3 + 610 \cdot 1 \\
    &= 5\boxed{911}
\end{align*}

\newpage

\item Let $S$ be the set of all complex numbers $w$ such that $|w|=53$. Let $a_1$ and $a_2$ be two distinct complex numbers in $S$ such that $ \displaystyle \frac{a_j-(7+13i)}{a_j-(55-7i)} $ is imaginary for $j=1,2$, and let $b_1$ and $b_2$ be two distinct complex numbers in $S$ such that $ \displaystyle \frac{b_k-(41+27i)}{b_k-(81+39i)} $  is imaginary for $k=1,2$. Let $z$ be the complex number that satisfies the property that both $\displaystyle \frac{z-a_1}{z-a_2}$ and $\displaystyle \frac{z-b_1}{z-b_2}$ are real. Then $z=m+ni$ for real numbers $m$ and $n$. Find $m+n$.

\textbf{Answer (68):} Denote the circle traced out by $|w| = 53$ by $\beta$. The condition on $a_1$ and $a_2$ is that $7+13i$, $a_j$, and $55-7i$ form a right triangle with the right angle at $a_j$, which is equivalent to the condition that $a_i$ lies on the circle with the diameter as the hypotenuse. The circle, which we will denote by $\xi$, has center $\frac{(7+13i)+(55-7i)}{2} = 31+3i$ and radius $|(7+13i)-(31+3i)| = \sqrt{24^2+10^2} = 26$. Similarly, the condition on $b_1$ and $b_2$ is that $41+27i$, $b_k$, and $81+39i$ form a right triangle with the right angle at $b_k$, which is equivalent to the condition that $b_k$ lies on the circle with the diameter as the hypotenuse. This circle, which we will denote by $\delta$, has center $\frac{(41+27i)+(81+39i)}{2} = 61+33i$ and radius $|(41+27i)-(61+33i)| = \sqrt{14^2+12^2} = 2\sqrt{109}$.

The conditions on $z$ mean that $a_1$, $a_2$, and $z$ are collinear, and $b_1$, $b_2$, and $z$ are collinear. Since $a_1$ and $a_2$ are the intersections of circles $\beta$ and $\xi$, the line through $a_1$ and $a_2$ is the radical axis of $\beta$ and $\xi$, and since $b_1$ and $b_2$ are the intersections of circles $\beta$ and $\delta$, the line through $b_1$ and $b_2$ is the radical axis of $\beta$ and $\delta$. So $z$ is the intersection of these two radical axes, and by the Radical Axis Theorem, the radical axis of $\xi$ and $\delta$ also passes through $z$. We then notice that the slope of the line passing through the centers of $\xi$ and $\delta$ is $1$, so the slope of the radical axis of $\xi$ and $\delta$, which is perpendicular, has slope $-1$, which means that all points on the radical axis will have the same sum of real and imaginary parts. This means that if we find a complex number on the radical axis, we will be done!

One such complex number on the radical axis we can find is the intersection of the radical axis with the line passing through the centers of $\xi$ and $\delta$, which we will denote $P$ and $Q$ respectively. Let $A$ be the intersection of $PQ$ with the radical axis, and let $B$ be one of the intersections of circles $\xi$ and $\delta$. Then $BP=26$, $BQ=2\sqrt{109}$, and $PQ=30\sqrt{2}$. By the Pythagorean theorem, $AP^2+AB^2=BP^2=676$ and $(30\sqrt{2}-AP)^2+AB^2=AQ^2+AB^2=BQ^2=436$. Subtracting the two equations gives $2 \cdot 30\sqrt{2} AP - (30\sqrt{2})^2 = 240$, and so $AP = \frac{240+1800}{60\sqrt{2}} = 17\sqrt{2}$. Therefore, $A$ as a complex number is $(31+3i)+(17+17i)=48+20i$, and the sum of the real and imaginary parts of $z$ is that of $A$, which is $48+20=\boxed{68}$.

\end{enumerate}

\iffalse

\fi

\end{document}
